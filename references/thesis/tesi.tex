\documentclass[oneside]{book}   

\usepackage{graphicx}

\usepackage{geometry}

\geometry{a4paper}



\begin{document}
	
	
	\begin{titlepage} 
		\noindent
		\begin{minipage}[t]{0.19\textwidth}
			\vspace{-4mm}{\includegraphics[scale=1.15]{logo_unimib.pdf}}
		\end{minipage}
		\begin{minipage}[t]{0.81\textwidth}
			{
				{\textsc{Università degli Studi di Milano - Bicocca}} \\
				\textbf{Scuola di Scienze} \\
				\textbf{Dipartimento di Informatica, Sistemistica e Comunicazione} \\
				\textbf{Corso di laurea in Informatica} \\
				\par
			}
		\end{minipage}
		\vspace{40mm}
		
		\begin{center}{\LARGE{\textbf{Enhancing irony detection with affective information}\par}}
		\end{center}        
		\vspace{50mm}
		\noindent
		{\large \textbf{Relatore:} Prof. Elisabetta Fersini} \\        
		\vspace{15mm}
		\begin{flushright}
			{\large \textbf{Relazione della prova finale di:}} \\
			\large{Gianluca Giudice} \\
			\large{Matricola 830694} 
		\end{flushright}
		\vspace{20mm}
		\begin{center}
			{\large{Anno Accademico 2019-2020}}
		\end{center}
	\end{titlepage}


\tableofcontents

% 3 PAGINE
\chapter*{Introduzione}

\section*{Descrizione del problema}

\section*{Approccio al problema}

\section*{Sintesi dei risultati}




\chapter{Stato dell'arte}

\section{Approccio supervisionato}

\section{Approccio non supervisionato}

\section{Approccio semi supervisionato}




\chapter{Sistema realizzato}

\section{Descrizine del sistema proposto}

\section{Rappresentazione del testo}

\subsection{Rappresentazione booleana}

\subsection{Rappresentazione mediante transformer}

\section{Caratteristiche linguistiche}
\subsection{PP}
\subsection{POS}
\subsection{Onomatopeic}
\subsection{...Tutto il resto}
\subsection{EMOT}

\section{Modelli supervisionati}
\subsection{Alberi di decisione}
\subsection{Support Vector Machine}
\subsection{Naive Bayes}
\subsection{Bayesian Network}

\section{Strumenti utilizzati}

\subsection{Scikit-learn}
\subsection{Weka}




\chapter{Campagna sperimentale}
\section{Dataset}
\section{Misure di performance}
\section{BOW + Caratteristiche linguistiche}
\section{BERT + Caratteristiche linguistiche}
\section{SBERT + Caratteristiche linguistiche}
\section{Analisi lessico con PCA}





\chapter{Conclusioni e sviluppi futuri}


\end{document}


TESI: 
- Introduzione (3 pagine)
- Descrizione problema
Comprensibile
Social network prodotto linguaggio naturale caratterizzato da ironia ed è importante saperla distinguure e il motivo.
- Come ho affronatato
Tenciche di NLP e problema di apprendiamento automatico usando machine learnign
- Sintesi risultati
- Struttura della tesi
- Capitolo 1 -> Stato dell'arte
1.1 approcci supervisoonati
1.2 non supervisionati
1.3 semi supervisionati
- Captiolo 2 -> Ssitema realizzato		
2.1 Descrizione del sitema proposto
Workflow e descrizione di massima del sistema
2.2 Rappresentazione del testo
2.2.1 Rappresentazione booleana
2.2.2 Rappresentazione mediante transformer
2.3 Caratteristiche linguistiche
2.3.1 POS
2.3.2 PP
2.3.3 Emot			
2.4 Modelli supervisionati
2.4.1 Alberi di decisione
ecc...
2.5 Strumenti utilizzati
2.5.1 SKleanr
2.5.2 Weka

- Capitolo 3 -> Campagna sperimetnale
3.0 Dataset e misure di performance
3.1 Rappresentazione BOW + caratteristiche linguistiche
3.2 Rappresentazione BERT + caratteristiche linguistiche
3.3 Analisi lessico con PCA
- Capitolo 4 -> Conlcusioni e sviluppi futuri
